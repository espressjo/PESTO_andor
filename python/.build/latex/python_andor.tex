%% Generated by Sphinx.
\def\sphinxdocclass{report}
\documentclass[letterpaper,10pt,english]{sphinxmanual}
\ifdefined\pdfpxdimen
   \let\sphinxpxdimen\pdfpxdimen\else\newdimen\sphinxpxdimen
\fi \sphinxpxdimen=.75bp\relax

\PassOptionsToPackage{warn}{textcomp}
\usepackage[utf8]{inputenc}
\ifdefined\DeclareUnicodeCharacter
 \ifdefined\DeclareUnicodeCharacterAsOptional
  \DeclareUnicodeCharacter{"00A0}{\nobreakspace}
  \DeclareUnicodeCharacter{"2500}{\sphinxunichar{2500}}
  \DeclareUnicodeCharacter{"2502}{\sphinxunichar{2502}}
  \DeclareUnicodeCharacter{"2514}{\sphinxunichar{2514}}
  \DeclareUnicodeCharacter{"251C}{\sphinxunichar{251C}}
  \DeclareUnicodeCharacter{"2572}{\textbackslash}
 \else
  \DeclareUnicodeCharacter{00A0}{\nobreakspace}
  \DeclareUnicodeCharacter{2500}{\sphinxunichar{2500}}
  \DeclareUnicodeCharacter{2502}{\sphinxunichar{2502}}
  \DeclareUnicodeCharacter{2514}{\sphinxunichar{2514}}
  \DeclareUnicodeCharacter{251C}{\sphinxunichar{251C}}
  \DeclareUnicodeCharacter{2572}{\textbackslash}
 \fi
\fi
\usepackage{cmap}
\usepackage[T1]{fontenc}
\usepackage{amsmath,amssymb,amstext}
\usepackage{babel}
\usepackage{times}
\usepackage[Bjarne]{fncychap}
\usepackage{sphinx}

\usepackage{geometry}

% Include hyperref last.
\usepackage{hyperref}
% Fix anchor placement for figures with captions.
\usepackage{hypcap}% it must be loaded after hyperref.
% Set up styles of URL: it should be placed after hyperref.
\urlstyle{same}

\addto\captionsenglish{\renewcommand{\figurename}{Fig.}}
\addto\captionsenglish{\renewcommand{\tablename}{Table}}
\addto\captionsenglish{\renewcommand{\literalblockname}{Listing}}

\addto\captionsenglish{\renewcommand{\literalblockcontinuedname}{continued from previous page}}
\addto\captionsenglish{\renewcommand{\literalblockcontinuesname}{continues on next page}}

\addto\extrasenglish{\def\pageautorefname{page}}

\setcounter{tocdepth}{1}



\title{python\_andor Documentation}
\date{May 21, 2019}
\release{1.0}
\author{Jonathan St-Antoine}
\newcommand{\sphinxlogo}{\vbox{}}
\renewcommand{\releasename}{Release}
\makeindex

\begin{document}

\maketitle
\sphinxtableofcontents
\phantomsection\label{\detokenize{index::doc}}

\index{command (class in python\_andor)}

\begin{fulllineitems}
\phantomsection\label{\detokenize{index:python_andor.command}}\pysigline{\sphinxbfcode{\sphinxupquote{class }}\sphinxcode{\sphinxupquote{python\_andor.}}\sphinxbfcode{\sphinxupquote{command}}}~\index{acq() (python\_andor.command method)}

\begin{fulllineitems}
\phantomsection\label{\detokenize{index:python_andor.command.acq}}\pysiglinewithargsret{\sphinxbfcode{\sphinxupquote{acq}}}{\emph{nbExp}, \emph{expTime}, \emph{**kwargs}}{}~\begin{description}
\item[{\sphinxstylestrong{Description}:}] \leavevmode
This function triggeres an acquisition sequence

\item[{\sphinxstylestrong{nbExp}: }] \leavevmode
number of exposure

\item[{\sphinxstylestrong{expTime}: }] \leavevmode
exposure time in seconds

\item[{\sphinxstylestrong{Options}:}] \leavevmode
\textendash{}nowoof\textendash{}: {[}True/False{]} Do not start the watch dog (does not block the main thread)

\item[{\sphinxstylestrong{Return}:}] \leavevmode
return void once the nb. of exposure is acquired

\end{description}

\end{fulllineitems}

\index{acq\_mode() (python\_andor.command method)}

\begin{fulllineitems}
\phantomsection\label{\detokenize{index:python_andor.command.acq_mode}}\pysiglinewithargsret{\sphinxbfcode{\sphinxupquote{acq\_mode}}}{\emph{mode}}{}~\begin{description}
\item[{\sphinxstylestrong{Description}:}] \leavevmode
This function sets the acquisition mode. Choices are : Target, video,flat,dark

\item[{\sphinxstylestrong{Return}:}] \leavevmode
return -1 if fails, void if successfull

\end{description}

\end{fulllineitems}

\index{initialize() (python\_andor.command method)}

\begin{fulllineitems}
\phantomsection\label{\detokenize{index:python_andor.command.initialize}}\pysiglinewithargsret{\sphinxbfcode{\sphinxupquote{initialize}}}{}{}~\begin{description}
\item[{\sphinxstylestrong{Description}:}] \leavevmode
Initialization sequences. This function will open a tclsh (version 8.5) console, then upload source client.tcl. Afterward a series of command will be sent to the window computer. Mostly, tcl source files will be uploaded.

\item[{\sphinxstylestrong{Return}:}] \leavevmode
void

\end{description}

\end{fulllineitems}

\index{script() (python\_andor.command method)}

\begin{fulllineitems}
\phantomsection\label{\detokenize{index:python_andor.command.script}}\pysiglinewithargsret{\sphinxbfcode{\sphinxupquote{script}}}{\emph{**kwargs}}{}~\begin{description}
\item[{\sphinxstylestrong{Description}:}] \leavevmode
This function will execute a script. The user will be prompt the enter the target name, exposure time and object name. The header, acquisition mode and the path will be automatically updated.

\item[{\sphinxstylestrong{Options}:}] \leavevmode
\textendash{}video\textendash{}: Not used interactivally. Will trigger the video function. Use the video function
\textendash{}no\_header: Used to not querry telinfo and telmeteo. If the script function fails the first time because of the telmeteo or telinfo, you can manually use the set\_header function and then use script(no\_header=True).

\item[{\sphinxstylestrong{Note}:}] \leavevmode\begin{enumerate}
\item {} 
Use the stop\_acq() function to stop the acquisition of th script.

\item {} 
the video flux will be automatically launched after exposuretime+2 secondes

\end{enumerate}

\end{description}

\end{fulllineitems}

\index{send() (python\_andor.command method)}

\begin{fulllineitems}
\phantomsection\label{\detokenize{index:python_andor.command.send}}\pysiglinewithargsret{\sphinxbfcode{\sphinxupquote{send}}}{\emph{cmd}}{}~\begin{description}
\item[{\sphinxstylestrong{Description}:}] \leavevmode
This function directly sends command to the window tclsh console. Do not use this function while another function is running

\end{description}

\end{fulllineitems}

\index{set\_header() (python\_andor.command method)}

\begin{fulllineitems}
\phantomsection\label{\detokenize{index:python_andor.command.set_header}}\pysiglinewithargsret{\sphinxbfcode{\sphinxupquote{set\_header}}}{}{}~\begin{description}
\item[{\sphinxstylestrong{Desceription}:}] \leavevmode
This function will set the header of the futur images. BonOMM must be open otherwise the function will fail.

\item[{\sphinxstylestrong{Return}:}] \leavevmode
This function returns 0 if successfull or -1 if unsuccessfull

\end{description}

\end{fulllineitems}

\index{set\_path() (python\_andor.command method)}

\begin{fulllineitems}
\phantomsection\label{\detokenize{index:python_andor.command.set_path}}\pysiglinewithargsret{\sphinxbfcode{\sphinxupquote{set\_path}}}{}{}~\begin{description}
\item[{\sphinxstylestrong{Description}:}] \leavevmode
This function will set the window working path. E.g., /190521/Target/test-omm/.

\item[{\sphinxstylestrong{Note}:}] \leavevmode
This function is useless if the class is used interactively

\item[{\sphinxstylestrong{Return}:}] \leavevmode
This function will return 0 if succesffull or -1 if it failed.

\end{description}

\end{fulllineitems}

\index{stop\_acq() (python\_andor.command method)}

\begin{fulllineitems}
\phantomsection\label{\detokenize{index:python_andor.command.stop_acq}}\pysiglinewithargsret{\sphinxbfcode{\sphinxupquote{stop\_acq}}}{}{}~\begin{description}
\item[{\sphinxstylestrong{Description}:}] \leavevmode
This function will stop a image acquisition

\end{description}

\end{fulllineitems}

\index{video() (python\_andor.command method)}

\begin{fulllineitems}
\phantomsection\label{\detokenize{index:python_andor.command.video}}\pysiglinewithargsret{\sphinxbfcode{\sphinxupquote{video}}}{\emph{**kwargs}}{}~\begin{description}
\item[{\sphinxstylestrong{Description}:}] \leavevmode
This function will start the video flux. The user will be ask to enter the exposure time

\item[{\sphinxstylestrong{Note}:}] \leavevmode\begin{enumerate}
\item {} 
use the stop\_acq() function to stop the video feed

\end{enumerate}

\end{description}

\end{fulllineitems}

\index{video\_flux() (python\_andor.command method)}

\begin{fulllineitems}
\phantomsection\label{\detokenize{index:python_andor.command.video_flux}}\pysiglinewithargsret{\sphinxbfcode{\sphinxupquote{video\_flux}}}{}{}~\begin{description}
\item[{\sphinxstylestrong{Description}:}] \leavevmode
This function starts the video flux

\item[{\sphinxstylestrong{Note}:}] \leavevmode
Normally this function is not used interactivelly

\end{description}

\end{fulllineitems}

\index{watch\_dog() (python\_andor.command method)}

\begin{fulllineitems}
\phantomsection\label{\detokenize{index:python_andor.command.watch_dog}}\pysiglinewithargsret{\sphinxbfcode{\sphinxupquote{watch\_dog}}}{\emph{expT}}{}~\begin{description}
\item[{\sphinxstylestrong{Description}:}] \leavevmode
This function starts the watch dog. Essentially, it launch an infinite loop that follows every image creation in the working path directory.

\item[{\sphinxstylestrong{Note}:}] \leavevmode
This function can be used interactivaly, but will block the main thread. Either use this function with the acq function or open another console the execute the stop\_acq function when you want the acquisition to stop.

\item[{\sphinxstylestrong{Return}:}] \leavevmode
void

\end{description}

\end{fulllineitems}


\end{fulllineitems}



\chapter{How to use python\_andor.py}
\label{\detokenize{start:how-to-use-python-andor-py}}\label{\detokenize{start::doc}}

\section{Getting Started}
\label{\detokenize{start:getting-started}}
\sphinxstylestrong{Getting the window comupter ready}
First, connect to the window desktop that is installed on the telescope with anydesk

\begin{figure}[htbp]
\centering
\capstart

\noindent\sphinxincludegraphics{{anydesk}.png}
\caption{Anydesk icon.}\label{\detokenize{start:id1}}\end{figure}

Then enter the number: 432 587 514. The password is Cinnamon\&Nutmeg. Then, launch Audel.

\begin{figure}[htbp]
\centering
\capstart

\noindent\sphinxincludegraphics{{audela}.png}
\caption{Audela icon.}\label{\detokenize{start:id2}}\end{figure}

In the console (see image), start the server script

\fvset{hllines={, ,}}%
\begin{sphinxVerbatim}[commandchars=\\\{\}]
\PYG{n+nb}{source} server.tcl
\end{sphinxVerbatim}

hello worl

\begin{figure}[htbp]
\centering
\capstart

\noindent\sphinxincludegraphics{{console}.png}
\caption{Audela consol.}\label{\detokenize{start:id3}}\end{figure}

First open a terminal (ctrl+alt+t in Ubuntu). Then enter

\fvset{hllines={, ,}}%
\begin{sphinxVerbatim}[commandchars=\\\{\}]
\PYG{k+kn}{from} \PYG{n+nn}{python\PYGZus{}andor} \PYG{k+kn}{import} \PYG{n}{command} \PYG{k}{as} \PYG{n}{com}

\PYG{n}{com} \PYG{o}{=} \PYG{n}{com}\PYG{p}{(}\PYG{p}{)}
\end{sphinxVerbatim}


\chapter{Indices and tables}
\label{\detokenize{index:indices-and-tables}}\begin{itemize}
\item {} 
\DUrole{xref,std,std-ref}{genindex}

\item {} 
\DUrole{xref,std,std-ref}{modindex}

\item {} 
\DUrole{xref,std,std-ref}{search}

\end{itemize}



\renewcommand{\indexname}{Index}
\printindex
\end{document}